\chapter{Evaluation} \label{evaluation}

What I want to see in this chapter:
\begin{itemize}
\item Table of all components
\item Table of synthesized programs with synthesis times for the different algorithms
\item Figure of the synthesis times of synthesized programs
\item Comparison to Feser
\item Explain why automatic black list and templates perform so poorly
\item Show advantages of using black lists (I know it's trivial)
\item Talk about the trivial thing that if you have more functions with the same time you will need much more time to find the program you are looking for.
\item Talk about the constants in the cost functions and how they affect the search space???
\item Stack versus queue expanding $\rightarrow$ mention it in the definitions
\end{itemize}

\section{Set up}
  \note{Describe the machine and the testing set up, which components were used, what information did you give to the synthesizer, how many examples were given. What else?}
  
\section{Cost functions}
  \note{Compare the different cost functions with each other, explain why are they good for some programs and bad for other. Table.}

\section{Black lists}
  \subsection{Benefits of black lists}
  \note{Table of your super manual black list. Figure that for each cost function compares time with and without black list. Trivial words about pruning search space and useless branches.}
  \subsection{Shortcomes of automatically generated black lists}
  \note{Maybe really bring the automatically generated table. Point at the repetitions. Say that automatically generated I/O-examples are bad and we need a lot of them. Say that you tried only identity pruning, but one could also try to generate, say, the empty list or whatever.}

\section{Templates}
  \note{Short section explaining that the templates you generate are not the templates you expect and why}


%%% Local Variables:
%%% mode: latex
%%% TeX-master: "thesis"
%%% End:
