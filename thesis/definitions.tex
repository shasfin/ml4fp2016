\lstset{style=plain}

\chapter{A type-driven synthesis procedure} \label{ch:definitions}

In this chapter we formally define our type-directed top-down synthesis procedure. We start with an intuitive description based on an example and move on to the formal definitions, starting with the target programming language and the search space. Finally, we present our synthesis procedure and some enhancements.

%Naturally, we would prefer smaller programs to larger ones, becuase they are easier for the user to inspect, but most importantly, because they are more likely to generalize the intended input-output behaviours outside of the examples specified by the user.  This is a crude instance of Occam's razor: from the multitude of possible explainations we prefer the one that is simplest\footnote{If it turns out that the found program does not have proper behavour beyond the given input-output examples, then the user can provide more examples to further guide the search.}.

%A crucial aspect in that we don't enumerate candidate programs blindly, but we take into account the type of the program to be synthesized, and also the types of the components in the library.  This is very important to reduce the space of candidate programs.

%The goal is to put together components from the library in a type-aware manner in order to get a list. More concretely, we fix \lstinline?X? to be a fixed input type variable, we fix \lstinline?n? to be an integer and \lstinline?x? to be a fixed input variable of type \lstinline?X?.

\section{The 'replicate' example}
Let us illustrate our synthesis procedure with the `replicate' example from Chapter~\ref{ch:introduction}.  We want to synthesise a program that takes a number $n$ and an element $x$, and returns the list $[x, \dots, x]$ that consists of $n$ copies of $x$.  A typical synthesis task would be specified like this:
\begin{lstlisting}[style=plain]
replicate :: $\forall$X. Int -> X -> List X
replicate 3  1 = [1,1,1]
replicate 2 [] = [[],[]]
\end{lstlisting}
Here, the user specifies the input-output signature of the program (its type), and a few input-output examples of its intended behaviour.  In this case, the user desires that the program can replicate elements of any type $X$, hence the type signature is prefixed with the $\forall X$ quantifier.  The ability to work with any type of elements is made evident by the two examples presented after the type signature.  In the first one, the number $1$ gets replicated three times; in the second one, the empty list \lstinline|[]| gets replicated twice.

In order to solve the synthesis task, we look for a program composed of components from a user-specified \emph{library of components}.  Let us assume that in our case the library consists of the standard list combinators \lstinline|map| and \lstinline|foldr|.  In addition, we include the \lstinline|enumTo| function that returns a list with the numbers from $1$ up to its argument, and also the \lstinline|const| function that always returns its first argument.  We will also need the standard list constructors \lstinline|cons| and \lstinline|[]|, and the standard integer constructors \lstinline|succ| and \lstinline|0|.

We search for the goal program by enumerating plausible programs, starting from simpler ones and moving to more complex ones (c.f., Occam's razor).  In the process, we test whether the enumerated programs meet the specified input-output behaviours.  Once we find such a program, we give it to the user for approval.  The main difficulty with this approach is that the number of candidate programs of a given size grows exponentially, but on the other hand very few of them make sense for the synthesis task.  For example, the program \lstinline|replicate n x = n| returns the integer $n$ instead of a list of elements, contradicting the specified type signature of `replicate'.

We address this problem by enumerating only well-typed programs that meet the user-specified type-signature.  We do that by actually enumerating \emph{partial programs}.  Partial programs are just programs with \emph{holes} that are to be filled-in later.  Every hole has a type that \emph{drives} subsequent search steps: they fill the hole with a partial program matching its type.  In our particular example, the search would start with the partial program:
\begin{lstlisting}[style=plain]
replicate n x = ?p
?p :: List X
\end{lstlisting}
The hole \lstinline|?p| is treated as a typed fresh symbol that has to be filled-in with another partial program.  Thus, our task is to find an instantiation for all holes recursively until we end up with a \emph{closed} program (all holes filled-in) that satisfies all the input-output examples provided by the user.

We structure the recursive enumeration process as a best-first search.  We maintain a set of current partial programs, the \emph{frontier}, whose holes are to be expanded.  At each step we select and remove a partial program of minimum \emph{cost} (e.g., size) from the frontier.  Then, we fill-in one of the holes in the selected program according to a set of rules.  The rules produce a set of \emph{successors} that are then added to the frontier.  The search continues until we select a closed program that meets all the given input-output examples.

The tricky part in the search are the rules that determine the successors of the selected program.  The rules must ensure that only well-typed programs are enumerated.  For example, in the above partial program we expand the lone hole \lstinline|?p|.  This hole has type \lstinline|List X| and we need to find a component in the library that is of this type.  This immediately excludes the integer constructors \lstinline|succ| and \lstinline|0|.  However, it also excludes \lstinline|enumTo|, as it produces a list of integers, while we need a list of \lstinline|X|, and \lstinline|X| could be any type, not necessarily \lstinline|Int|.  All other possibilities are open: we can fill \lstinline|?p| with the result of a \lstinline|fold|, a \lstinline|map|, a \lstinline|const|, or with the empty list \lstinline|[]|.  Let us begin with \lstinline|[]|:
\begin{lstlisting}[style=plain]
replicate n x = []
?p = [] :: List X
\end{lstlisting}

Since this program is closed, we can test it on the input-output examples, but it satisfies none of them.  Therefore, we consider the other four possible instantiations of \lstinline!?p! as well:
\begin{lstlisting}[style=plain]
replicate n x = cons ?x ?xs
?p = cons ?x ?xs :: List X
?x :: X
?xs :: List X
\end{lstlisting}
\begin{lstlisting}[style=plain]
replicate n x = foldr ?f ?init ?xs
?p = foldr ?f ?init ?xs :: List X
?f :: ?Y -> List X -> List X
?init :: List X
?xs :: List ?Y
\end{lstlisting}
\begin{lstlisting}[style=plain]
replicate n x = const ?xs ?s
?p = const ?xs ?s :: List X
?xs :: List X
?s :: ?Y
\end{lstlisting}
\begin{lstlisting}[style=plain]
replicate n x = map ?f ?xs
?p = map ?f ?xs :: List X
?f :: ?Y -> X
?xs :: List ?Y
\end{lstlisting}
In these programs, \lstinline!?Y! is a fresh type variable that will be instantiated later.  It indicates that we do not know the type of the first argument of \lstinline!?f!. The only thing we know is that it has to match the type of the elements of \lstinline!?xs!.

The next step in the procedure is to expand some hole in the program of least cost.  In Section~\ref{Cost functions} we discuss several choices for a cost function, but here we just select programs that will lead us to the solution.  Let us select the last program for expansion, i.e., \lstinline|map ?f ?xs|.  We have two holes to fill-in: a function \lstinline!?f! that takes a \lstinline|?Y| and returns an \lstinline|X|, and a list \lstinline!?xs! of \lstinline|?Y|.  We decide to expand the hole \lstinline|?f| first.  Obviously, for that we cannot use \lstinline|map| or \lstinline|enumTo|, because they return lists, whereas \lstinline|?f| must return an \lstinline|X|.  The other two possibilities are \lstinline|foldr| and \lstinline|const|, which we add the to the frontier:

\begin{lstlisting}[style=plain]
replicate n x = map (foldr ?g ?init) ?xs
?p = map ?f ?xs :: List X
?f = foldr ?g ?init :: List ?Z -> X
?g :: ?Z -> X -> X
?init :: X
?xs :: List (List ?Z)
\end{lstlisting}

\begin{lstlisting}[style=plain]
replicate n x = map (const ?x) ?xs
?f = const ?x :: ?Y -> X
?x :: X
?xs :: List ?Y
\end{lstlisting}
Note that in the first case we instantiated \lstinline|?Y| with \lstinline|List ?Z|, because \lstinline|foldr|takes a list as its last argument.  This indicates that type-instantiation is a non-trivial task.  In particular, it requires solving \emph{unification} constraints between arguments and return values, as discussed in Section~\ref{Unification}.

Now, we need to expand a program in the frontier again.  Let us assume that the program of least cost is \lstinline|map (const ?x) ?xs|.  There are two holes to fill-in: \lstinline|?x| of type \lstinline|X| and \lstinline|?xs| of type \lstinline|List ?Y|.  Interestingly, we have only one option for the first hole, and so we take it: \lstinline|replicate|'s second argument \lstinline|x|:
\begin{lstlisting}[style=plain]
replicate n x = map (const x) ?xs
?p = map ?f ?xs :: List X
?f = const ?x :: ?Y -> X
?x = x :: X
?xs :: List ?Y
\end{lstlisting}

We select this newly added program from the frontier, and fill-in its only hole \lstinline|?xs|.  We are now in a situation similar to where the whole search started: we have to generate a list.  However, this time the type of the elements is not fixed, and we cannot rule out \lstinline|enumTo|.  Therefore we have a lot of possibilities to instantiate this hole, starting with \lstinline|[]| and ending with \lstinline|enumTo ?n| where \lstinline|?n| is a fresh hole of type \lstinline|Int|.

One of the candidate programs, \lstinline|map (const x) []|, is closed, and we evaluate it on the input-output examples.  However, this program does not satisfy any of them, thus we rule it out.  Several candidate solutions are added to the frontier at this step, but let us focus on the most promising one:
\begin{lstlisting}[style=plain]
replicate n x = map (const x) (enumTo ?n)
?p = map ?f ?xs :: List X
?f = const ?x :: ?Y -> X
?x = x :: X
?xs = enumTo ?n :: List Int
?n :: Int.
\end{lstlisting}

The only hole to expand is \lstinline|?n| and has type \lstinline|Int|.  Again, we have a lot of possibilities: the number \lstinline|0|, \lstinline|replicate|'s first argument \lstinline|n|, the constructor \lstinline|succ| applied to another integer hole, or the result of invoking \lstinline|const|.  We again look at the closed successors, namely the first two:
\begin{lstlisting}
replicate n x = map (const x) (enumTo 0)
\end{lstlisting}
\begin{lstlisting}
replicate n x = map (const x) (enumTo n)
\end{lstlisting}
The first successor does not satisfy the input-output examples.  On the other hand, the second one does, and we finally arrived at a solution.  Further inspection by the user shows that this program indeed captures the desired `replicate' functionality.  Our synthesis task is complete.

This example showed how to use type-directed best-first search to generate a program with specified input-output behaviour, given a library of basic components.  The many search choices that we left unexplored indicate that the search space remains quite big even if with the type-directed search that rules out ill-typed programs.  In Section~\ref{Black list} we show an additional way to rule out superfluous programs from the search space, by blacklisting hole instantiations that have the same (or similar) input-output behaviour.

\subsection{Summary}
\begin{description}
\item[Hole] unknown part of a program that can be instantiated with some other programs. Only its type is known.
\item[Closed program] a program without holes that can be evaluated on the input-output examples. The terms and the types of our calculus are formally defined in Section~\ref{Term and types}.
\item[Type-aware expansion of holes] we expand holes based on their type. In Section~\ref{Search space} we can find the rules according to which a program is expanded.
\item[Best first search] a frontier of programs with holes is maintained, one hole of the least cost program is expanded in every iteration. The best first search algorithm is defined in Section~\ref{Exploration} and some cost functions are presented in Section~\ref{Cost functions}.
\item[Superfluous program] a program that is equivalent to a shorter program.
\end{description}

 
\section{Calculus}

The programs targeted by our synthesiser are based on System F terms restricted to applications of library components and input variables.  As we saw in the previous example, during synthesis an additional syntactic construct is used: the hole.  We denote the language based on application of library components, input variables and holes as the \emph{target language} of our synthesiser.
However, we also need another language: the language in which the library components are defined and the programs are evaluated on the input-output examples.  In the rest of the chapter we will call this language the \emph{internal language}.  It extends System F with holes, parametric types and recursion.

We present therefore in Section~\ref{Term and types} the syntax of the following three calculi:
\begin{enumerate}[1.]
\item System F, which we provide for the sake of completeness, since the other two calculi build upon it;
\item The internal language: an extension of System F with holes, input variables, library components, parametric types and recursive terms and types;
\item The target language: a subset of the internal language, featuring only application of components, holes and input variables.
\end{enumerate}
The notation and the exposition closely follow the excellent book on type systems by Benjamin Pierce \cite{pierce2002types}.  We refer to the book for a thorough introduction to System F and to type systems in general.

  \subsection{Terms and Types}\label{Term and types}
This section presents the syntax of three different calculi: System F, the internal language and the target language.

\paragraph{System F} System F, also known as the polymorphic lambda calculus, is a calculus that, additionally to term abstraction and term application, features two new kinds of terms: type abstraction \lstinline!$\Lambda$X. t! and type application \lstinline!t [T]!. This allows us to express polymorphic functions. For example, the polymorphic identity function is defined as \lstinline!$\Lambda$X. $\lambda$x:X. x!.
Polymorphic functions, defined as type abstractions, have a special type: the \emph{universal} type \lstinline!$\forall$X. T!. For a more detailed introduction to System F we refer to \cite{pierce2002types}. The syntax is summarized below.

 \begin{plstx}
(terms): t ::= x | \lambda x : T.\; t | t\;t | \Lambda X.\; t | t\;[T]\\
(types): T ::= X | T \rightarrow T | \forall X.\; T\\
(variable bindings): \Gamma ::= \emptyset | \Gamma \cup \{x : T\} | \Gamma \cup \{X\}\\
\end{plstx}


\paragraph{Internal language} We extend System F with holes $?x$, input variables $i$ as well as named library components $c$ and named types $C$ that can take parameters $C\; T_1\;\ldots\; T_K$. The number of type parameters supported by a named type is denoted as $K$ in its definition. The use of the names enables recursion in the definition of library components and types. Terms that do not contain holes are called \emph{closed}.
The syntax of our calculus is summarised below. Evaluation and typing rules for this calculus can be found in the respective subsections.

 \begin{plstx}
(terms): t ::= x | \lambda x : T.\; t | t\;t | \Lambda X.\; t | t\;[T] | c | {?x} | i\\
(types): T ::= X | T \rightarrow T | \forall X.\; T | ?X | I | C\; T\;\ldots\; T\\
(variable bindings): \Gamma ::= \emptyset | \Gamma \cup \{x : T\} | \Gamma \cup \{X\}\\
(hole bindings): \Xi ::= \emptyset | \Xi \cup \{{?x} : T\} | \Xi \cup \{{?X}\}\\
(input variable bindings): \Phi ::= \emptyset | \Phi \cup \{i = t : T\} | \Phi \cup \{I = T\}\\
(library components): \Delta ::= \emptyset | \Delta \cup \{c = t : T\} | \Delta \cup \{C = T : K\}\\
\end{plstx}
Note that we have three additional contexts.
The first one, $\Xi$, collects type and term holes. Moreover, it binds term holes to their types.
The second one, $\Phi$, is the library of input variables. It contains one concrete instantiation of the input variables. It binds a definition and a type signature to each input term variable and a definition to each input type variable.
The third one, $\Delta$, is the library of components. Each named term is bound to its definition and to its type signature and each named type is bound to its definition and to the number of parameters it takes.


\paragraph{Target language} The target language of our synthesiser is a subset of the internal language. We already saw terms of this language in the `replicate' example, but there all type applications were omitted for the sake of clarity.
Formally, the target language restricts the internal language to term and type application of library components, input variables and holes as follows.
 \begin{plstx}
(terms): t ::= t\;t | t\;[T] | c | ?x | i\\
(types): T ::= X | T \rightarrow T | \forall X.\; T | ?X | I | C\; T\;\ldots\; T\\
(hole bindings): \Xi ::= \emptyset | \Xi \cup \{{?x} : T\} | \Xi \cup \{{?X}\}\\
(input variable bindings): \Phi ::= \emptyset | \Phi \cup \{i = t : T\} | \Phi \cup \{I = T : K\}\\
(library components): \Delta ::= \emptyset | \Delta \cup \{c = t : T\} | \Delta \cup \{C = T : K\}\\
\end{plstx}
We do not need separate typing rules for this sublanguage of the internal language, since it is closed under the typing rules. We omit evaluation rules for this calculus, as our evaluator is based on the internal language.

\paragraph{Program} A program is defined as the 4-tuple $\{\Xi, \Phi, \Delta \vdash t :: T\}$, where $t$ is a term of the target language. A program is called \emph{closed} if $\Xi$ is empty and $t$ and $T$ do not contain holes.


  \subsection{Encodings}\label{Encodings}
Familiar types such as booleans, integers or lists do not appear in the definition of the types of the internal language. All these types can be encoded in the type system of the internal language using either Church's or Scott's encoding \cite{ScottNumerals}. We opt for Scott's encoding because it is more efficient in our case.

Scott's booleans coincide with Church's booleans and are encoded as follows.
\begin{lstlisting}[style=plain, mathescape]
Bool  = $\forall$R. R $\rightarrow$ R $\rightarrow$ R
true  = $\Lambda$R. $\lambda x_1$:R. $\lambda x_2$:R. $x_1$
      : Bool
false = $\Lambda$R. $\lambda x_1$:R. $\lambda x_2$:R. $x_2$
      : Bool
if-then-else = $\Lambda$X. $\lambda$b:Bool. $\lambda$t:X. $\lambda$f:X. b [X] t f
             : $\forall$X. Bool $\rightarrow$ X $\rightarrow$ X $\rightarrow$ X
\end{lstlisting}

Scott's integers differ from Church's integers as they unwrap the constructor only once. Therefore they are more suitable for pattern matching.
\begin{lstlisting}[style=plain, mathescape]
Int = $\forall$R. R $\rightarrow$ (Int $\rightarrow$ R) $\rightarrow$ R
zero = $\Lambda$R. $\lambda$z:R. $\lambda$s:Int $\rightarrow$ R. z
     : Int
succ = $\lambda$n:Int. $\Lambda$R. $\lambda$z:R. $\lambda$s:Int $\rightarrow$ R. s n
     : Int $\rightarrow$ Int
case = $\Lambda$R. $\lambda$n:Int. $\lambda$a:R. $\lambda$f:Int $\rightarrow$ R. n [R] a f
     : $\forall$R. Int $\rightarrow$ R $\rightarrow$ (Int $\rightarrow$ R) $\rightarrow$ R
\end{lstlisting}

Analogously to integers, Scott's lists are a recursive type and naturally support pattern matching.
\begin{lstlisting}[style=plain, mathescape]
List X = $\forall$R. R $\rightarrow$ (X $\rightarrow$ List X $\rightarrow$ R) $\rightarrow$ R
nil = $\Lambda$X. $\Lambda$R. $\lambda$n:R. $\lambda$c:X $\rightarrow$ List X $\rightarrow$ R. n
    : $\forall$X. List X
con = $\Lambda$X. $\lambda$x:X. $\lambda$xs:List X. $\Lambda$R. $\lambda$n:R. $\lambda$c:X $\rightarrow$ List X $\rightarrow$ R. c x xs
    : $\forall$X. X $\rightarrow$ List X $\rightarrow$ List X
case = $\Lambda$X. $\Lambda$Y. $\lambda$l:List X. $\lambda$n:Y. $\lambda$c:X $\rightarrow$ List X $\rightarrow$ Y. l [Y] n c
     : $\forall$X. $\forall$Y. List X $\rightarrow$ Y $\rightarrow$ (X $\rightarrow$ List X $\rightarrow$ Y) $\rightarrow$ Y
\end{lstlisting} 

Other algebraic datatypes, such as trees, can be easily encoded in an analogous manner.

  \subsection{Evaluation semantics}\label{Evaluation}
In this section we present the evaluation semantics of our internal language, that is the second calculus introduced in Section~\ref{Term and types}. The evaluation semantics is a standard eager evaluation and we refer to the excellent book by Benjamin Pierce about type systems \cite{pierce2002types} for an introduction to the evaluation semantics of System F and evaluation rules in general.

The evaluation judgement $\Phi, \Delta \vdash t \longrightarrow t'$ means that the term $t$ evaluates in one step to the term $t'$ under the free variable bindings library $\Phi$, that contains concrete instantiations for the input variables, and the component library $\Delta$, that contains the definitions of the library components. Before listing the evaluation rules, let us define \emph{value} $v$ to be a term to which no evaluation rules apply.

\begin{prooftree}
\AxiomC{$c = t : T \in \Delta$}
	\RightLabel{E-Lib}
	\UnaryInfC{$\Phi, \Delta \vdash c \longrightarrow t$}
\end{prooftree}

\begin{prooftree}
\AxiomC{$i = t : T \in \Phi$}
	\RightLabel{E-Inp}
	\UnaryInfC{$\Phi, \Delta \vdash i \longrightarrow t$}
\end{prooftree}

\begin{prooftree}
\AxiomC{$\Phi, \Delta \vdash t_1 \longrightarrow t_1'$}
	\RightLabel{E-App1}
	\UnaryInfC{$\Phi, \Delta \vdash t_1\; t_2 \longrightarrow t_1'\; t_2$}
\end{prooftree}

\begin{prooftree}
\AxiomC{$\Phi, \Delta \vdash t_2 \longrightarrow t_2'$}
	\RightLabel{E-App2}
	\UnaryInfC{$\Phi, \Delta \vdash v_1\; t_2 \longrightarrow v_1\; t_2'$}
\end{prooftree}

\begin{prooftree}
\AxiomC{}
	\RightLabel{E-AppAbs}
	\UnaryInfC{$\Phi, \Delta \vdash (\lambda x:T_{11}.\; t_{12})\; v_2 \longrightarrow [x \mapsto v_2]t_{12}$}
\end{prooftree}

\begin{prooftree}
\AxiomC{}
	\RightLabel{\textsc{E-AppAbs}}
	\UnaryInfC{$\Phi, \Delta \vdash (\Lambda X.\; t_2)\; [T_2] \longrightarrow [X \mapsto T_2]t_2$}
\end{prooftree}  
  
Rules E-Lib and E-Inp load the definitions of library components or input variables from the respective library. E-App1 and E-App2 evaluate the left hand side, respectively the right hand side, of a term application. E-AppAbs and \textsc{E-AppAbs} get rid of a term and, respectively, type abstraction and substitute the argument into the body.
Note that E-App2 applies only if the left hand side of the application cannot be evaluated further and that E-AppAbs applies only when the argument of the lambda abstraction is a value, determining the order of evaluation.

\subsection{Type checking}\label{Typing}

In this section we will present the typing rules of the internal language, that is the second calculus presented in Section~\ref{Term and types}. The typing judgement $\Gamma, \Xi, \Phi, \Delta \vdash t : T$ means the term $t$ has type $T$ in the contexts $\Gamma$ and $\Xi$, binding respectively variables and holes, and $\Phi$ and $\Delta$, containing signatures and definitions of respectively input variables and library components. The typing judgement is similar to the typing judgement of System F. As usual, we refer to \cite{pierce2002types} for more details.
  
\begin{prooftree}
\AxiomC{$x : T \in \Gamma$}
	\RightLabel{T-Var}
	\UnaryInfC{$\Gamma, \Xi, \Phi, \Delta \vdash x : T$}
\end{prooftree}

\begin{prooftree}
\AxiomC{${?x} : T \in \Xi$}
	\RightLabel{T-Hol}
	\UnaryInfC{$\Gamma, \Xi, \Phi, \Delta \vdash {?x} : T$}
\end{prooftree}

\begin{prooftree}
\AxiomC{$i = t : T \in \Phi$}
	\RightLabel{T-Inp}
	\UnaryInfC{$\Gamma, \Xi, \Phi, \Delta \vdash i : T$}
\end{prooftree}

\begin{prooftree}
\AxiomC{$c = t : T \in \Delta$}
	\RightLabel{T-Lib}
	\UnaryInfC{$\Gamma, \Xi, \Phi, \Delta \vdash c : T$}
\end{prooftree}

\begin{prooftree}
\AxiomC{$\Gamma \cup \{x : T_1\}, \Xi, \Phi, \Delta \vdash t_2 : T_2$}
	\RightLabel{T-Abs}
	\UnaryInfC{$\Gamma, \Xi, \Phi, \Delta \vdash \lambda x : T_1.\; t_2 : T_1 \rightarrow T_2$}
\end{prooftree}

\begin{prooftree}
\AxiomC{$\Gamma, \Xi, \Phi, \Delta \vdash t_1 : T_1 \rightarrow T_2$}
\AxiomC{$\Gamma, \Xi, \Phi, \Delta \vdash t_2 : T_1$}
	\RightLabel{T-App}
	\BinaryInfC{$\Gamma, \Xi, \Phi, \Delta \vdash t_1\;t_2 : T_2$}
\end{prooftree}

\begin{prooftree}
\AxiomC{$\Gamma \cup \{X\}, \Xi, \Phi, \Delta \vdash t_2 : T_2$}
	\RightLabel{\textsc{T-Abs}}
	\UnaryInfC{$\Gamma, \Xi, \Phi, \Delta \vdash \Lambda X.\; t_2 : \forall X.\; T_2$}
\end{prooftree}

\begin{prooftree}
\AxiomC{$\Gamma, \Xi, \Phi, \Delta \vdash t_1 : \forall X.\; T_{12}$}
	\RightLabel{\textsc{T-App}}
	\UnaryInfC{$\Gamma, \Xi, \Phi, \Delta \vdash t_1\;[T_2] : [X \mapsto T_2]T_{12}$}
\end{prooftree}

The rules T-Var, T-Hol, T-Inp and T-Lib load the signature of variables, holes, input variables and library components respectively from their contexts.
T-Abs types a lambda abstraction as an arrow type from the type of its variable to the type of its body. \textsc{T-Abs} gives a type abstraction a universal type in accordance to the type of its body.
An application typechecks only if the left hand side has an arrow type and the type of the argument is equal to the type of the argument of the left hand side. Analogously, a type application typechecks only if the left hand side has a universal type.

\subsection{Type unification}\label{Unification}

In order to identify the library components that can be used to instantiate a hole of a given type, we need type unification. Consider, for example, a hole of type \lstinline?List Int -> Int?. We want to instantiate this hole not only with components that have precisely this type, like \lstinline?sum? and \lstinline?prod?, but also components with a more general type that can be matched to the desired type through type application, like \lstinline?head :: $\forall$X. List X -> X? and \lstinline?length :: $\forall$X. List X -> Int?.

However, unification on the type system of System F is undecidable \cite{Huet1975}.
Therefore we choose to restrict our type system to quantifier-free forms.
This allows us to completely ignore universal types during unification. Nonetheless, we still want to handle universally quantified library components, that is components whose type has the form \lstinline?$\forall X_1$. $\ldots$ $\forall X_n$. T($X_1, \ldots, X_n$)? with a quantifier-free \lstinline?T($X_1, \ldots, X_n$)?. Towards this end, we represent types of the form \lstinline?$\forall X_1$. $\ldots$ $\forall X_n$. T($X_1, \ldots, X_n$)? as \lstinline!T(?$X_1$, $\ldots$, ?$X_n$)!, that is we leave out all quantifiers and replace the bound variables with fresh type holes.

The goal of unification is to find a substitution $\sigma$ that unifies all pairs of types of a set of constraints.
A set of constraints $\C$ is a set containing pairs of types $(S, T)$ that should be equal under a \emph{substitution} (a mapping from holes to types). That is, for the output of the unification algorithm $\sigma$ it must hold $\sigma(S) = \sigma(T)$ for every constraint $(S, T)$ in $\C$.

Our unification algorithm (summarised as Algorithm~\ref{alg:type unification} below) is based on the unification algorithm for typed lambda calculus from \cite{pierce2002types} and slightly modified to fit our needs.

A type hole unifies with anything. An arrow type \lstinline?$T_1$ -> $T_2$? unifies either with a type hole or with another arrow type \lstinline?$T_3$ -> $T_4$? if \lstinline?$T_1$? unifies with \lstinline?$T_3$? and \lstinline?$T_2$? with \lstinline?$T_4$?. A named type applied to all of its parameters \lstinline?C $T_{11}$ $\ldots$ $T_{1k}$? unifies either with a type hole or with the same named type applied to the same number of parameters \lstinline?C $T_{21}$ $\ldots$ $T_{2k}$? if the respective parameters \lstinline?$T_{1j}$? and \lstinline?$T_{2j}$? unify for all $j = 1, \ldots, k$. Universal  types unify only with type holes and should not appear in the set of constraints.

\begin{algorithm}
\caption{Type unification\label{alg:type unification}}
\KwIn{Set of constraints $\C = \{(T_{11}, T_{12}), (T_{21}, T_{22}), \ldots\}$}
\KwOut{Substitution $\sigma$ so that $\sigma(T_{i1}) = \sigma(T_{i2})$ for every constraint $(T_{i1}, T_{i2})$ in $\C$}

\SetKwProg{Fn}{Function}{ is}{end}
\Fn{unify($\C$)}{
	\DontPrintSemicolon
	\lIf{$\C = \emptyset$}{
		[]
	}\Else{let $\{(T_1, T_2)\} \cup \C' = \C$ in\\
		\uIf{$T_1 = T_2$}{
			\textit{unify}$(\C')$
		}\uElseIf{$T_1 = {?X}$ and ${?X}$ does not occur in $T_2$}{
			\textit{unify}$([?X \mapsto T_2]\C') \circ [?X \mapsto T_2]$
		}\uElseIf{$T_2 = {?X}$ and ${?X}$ does not occur in $T_1$}{
			\textit{unify}$([?X \mapsto T_1]\C') \circ [?X \mapsto T_1]$
		}\uElseIf{$T_1 = T_{11} \rightarrow T_{12}$ and $T_2 = T_{21} \rightarrow T_{22}$}{
			\textit{unify}$(\C' \cup \{T_{11} = T_{21}, T_{12} = T_{22}\})$
		}\uElseIf{$T_1 = C\;T_{11}\; T_{12}\; \ldots\; T_{1k}$ and $T_2 = C\; T_{21}\; T_{22}\; \ldots,\; T_{2k}$}{
			\textit{unify}$(\C' \cup \{T_{11} = T_{21}, T_{12} = T_{22}, \ldots, T_{1k} = T_{2k}\})$
		}\Else{
			\textit{fail}
		}
	}
}
\end{algorithm} 

\section{Search}
After defining the target language, the evaluation semantics, the type checking and the type unification, we are ready to formally define the problem, the search space and the synthesis procedure.

\paragraph{Problem definition} Given a library $\Delta$, a goal type $T$ and a list of input-output examples $[(\Phi_1, o_1), \ldots , (\Phi_N, o_N)]$, find a closed term $t$ in the target language such that
\begin{enumerate}[(i)]
\item the abstraction of $t$ over all of its type and term input variables has the goal type under an empty variable binding context and an empty hole binding context, that is $\emptyset, \emptyset, \Phi_1, \Delta \vdash t' : T$ where $t'$ is \[\Lambda X_1.\; \ldots\; \Lambda X_j.\; \lambda x_1.\; \ldots\; \lambda x_k.\; [I_1 \mapsto X_1, \ldots, I_j \mapsto X_j, i_1 \mapsto x_1, \ldots, x_k \mapsto x_k]t.\]
\item $t$ satisfies all input-output examples, that is $\Phi_n, \Delta \vdash t \longrightarrow^* t'$ and $\Phi_n, \Delta \vdash o_n \longrightarrow^* t'$ for all $n = 1, \ldots, N$.
\end{enumerate}

In Section~\ref{Search space} we define the search space and in Section~\ref{Exploration} we describe the main enumeration algorithm, a standard best-first search.

\subsection{Search space}\label{Search space}
The search space is structured as a graph, where the vertices correspond to \emph{programs}. Recall that a program is the 4-tuple $\{\Xi, \Phi, \Delta \vdash t :: T'\}$, where $t$ is a term of the target language. The type $T'$ can be transformed into the goal type $T$ abstracting over all type and term input variables. That is, if $\Phi$ contains the type input variables $I_1, \ldots, I_j$ and the signatures of the term input variables $i_1 : T_1, \ldots, i_k : T_k$, then the goal type $T$ should be equal to
\[\forall X_1.\; \ldots \forall X_j.\; [I_1 \mapsto X_1, \ldots, I_j \mapsto X_j] (T_1 \rightarrow \ldots \rightarrow T_k \rightarrow T').\]

There is a directed edge between two programs $\{\Xi_1, \Phi, \Delta \vdash t_1 :: T'\}$ and $\{\Xi_2, \Phi, \Delta \vdash t_2 :: T'\}$ if and only if the judgement \emph{derive} (defined below) $\Xi, \Phi, \Delta \vdash t_1 :: T_1 \Mapsto \Xi', \Phi, \Delta \vdash t_2 :: T_2$ holds between the two.

To express the rules of the derive judgement in a more compact form, we introduce \emph{evaluation contexts}. An evaluation context is an expression with exactly one syntactic hole $[]$ into which we can plug any term. For example, if we have the context $\mathcal{E}$ we can place the term $t$ into its hole and denote this new term by $\mathcal{E}[t]$.

The derive judgement can be summarised in four rules.
D-VarLib replaces a hole $?x$ with a type application of a library component $c$ to suitable types, if the type of $c$ unifies with the type of $?x$. D-VarInp replaces a hole $?x$ with an input variable $i$ from the context $\Phi$, if its type unifies with the type of the hole. D-VarApp turns a hole into a term application of two fresh holes. The rule D-App chooses a hole in the program and expands it according to one of the three rules above.

The notation $\sigma(\Xi)$ denotes the application of the substitution $\sigma$ to all types appearing in $\Xi$.

\begin{prooftree}
\AxiomC{$c : \forall X_1.\; \ldots \forall X_n.\; T_c(X_1, \ldots, X_n) \in \Delta$}
\noLine
\UnaryInfC{$?X_1, \ldots, ?X_n$ are fresh type holes}
\noLine
\UnaryInfC{$\sigma$ unifies $T$ with $T_c(?X_1, \ldots, ?X_n)$}
\noLine
\UnaryInfC{$\Xi' = \Xi \cup \{{?X_1}, \ldots, {?X_n}\} \setminus \{{?x} : T\}$}
	\RightLabel{D-VarLib}
	\UnaryInfC{$\Xi, \Phi, \Delta \vdash {?x} :: T \Mapsto \sigma(\Xi'), \Phi, \Delta \vdash c\; [\sigma(?X_1)]\; \ldots\; [\sigma(?X_n)] :: \sigma(T) $
	}
\end{prooftree}

\begin{prooftree}
\AxiomC{$i : T_i \in \Phi$}
\AxiomC{$\sigma$ unifies $T$ with $T_i$}
	\RightLabel{D-VarInp}
	\BinaryInfC{$\Xi, \Phi, \Delta \vdash {?x} :: T \Mapsto \sigma(\Xi \setminus \{{?x} : T\}), \Phi, \Delta \vdash i :: \sigma(T) $
	}
\end{prooftree}


\begin{prooftree}
\AxiomC{$?X$ is a fresh type variable}
\noLine
\UnaryInfC{$\Xi' = {\Xi \setminus \{?x:T\} \cup \{{?x_1} : {?X} \rightarrow T, {?x_2} : {?X}, {?X}\}}$}
	\RightLabel{D-VarApp}
	\UnaryInfC{$\Xi, \Phi, \Delta \vdash {?x} :: T \Mapsto \Xi', \Phi, \Delta \vdash {?x_1}\;{?x_2} :: T$}
\end{prooftree}


\begin{prooftree}
\AxiomC{$\Xi, \Phi, \Delta \vdash {?x} :: T_1 \Mapsto \Xi', \Phi, \Delta \vdash t_1' :: T_1'$}
	\RightLabel{D-App}
	\UnaryInfC{$\Xi, \Phi, \Delta \vdash t[{?x}] :: T \Mapsto \Xi', \Phi, \Delta \vdash t[t_1] :: [T_1 \mapsto T_1']T$}
\end{prooftree}

All derived programs are well-typed and, since the types of all derived programs unify with the types of their ancestors, have the desired type.

\subsection{Best-first search}\label{Exploration}
Our enumeration procedure traverses the search graph defined in the previous section using a standard best-first search. The algorithm maintains a frontier of candidate programs and expands one hole of the most promising program in each iteration. Closed programs are evaluated on the input-output examples. The search terminates when the first program that satisfies all input-output examples is found.

Algorithm~\ref{alg:best-first search} is parametrised with respect to the following two questions.
\begin{enumerate}
\item Which candidate program is the most promising?
\item Which hole of the most promising program should be expanded?
\end{enumerate}
The first question is addressed by the implementation of the \lstinline?compare? function over programs. We compare programs based on their \emph{cost}. Section~\ref{Cost functions} presents several cost functions.

The second question is addressed by the implementation of the \lstinline?successor? function. In particular, the implementation of the D-App rule. There are two easy ways to handle this situation. The first way is to always expand the leftmost hole, the second way is to always expand the oldest hole. Section~\ref{Stack vs Queue} discusses the advantages and disadvantages of these two choices.

\begin{algorithm}
\caption{Best first search\label{alg:best-first search}}
\KwIn{goal type $T$, library components $\Delta$, list of input-output examples $[(\Phi_1, o_1), \ldots , (\Phi_N, o_N)]$}
\KwOut{closed program $\{\Xi, \Phi_1, \Delta \vdash t :: T\}$ that satisfies all I/O-examples}

queue $\gets$ PriorityQueue.empty {\color{blue}compare}\\
queue $\gets$ PriorityQueue.push queue $\{\Xi, \Phi_1, \Delta \vdash {?x} :: T\}$\\

\While{not ((PriorityQueue.top queue) satisfies all I/O-examples)}{
	successors $\gets$ {\color{blue}successor} (PriorityQueue.top queue)\\
	queue $\gets$ PriorityQueue.pop queue\\
	\For{all s in successors}{
		queue $\gets$ PrioriryQueue.push queue s\\
	}
}
\Return{PriorityQueue.top queue}
\end{algorithm}

\section{Cost functions}\label{Cost functions}
The compare function in the best-first search algorithm can be defined as \lstinline?cost $p_1$ - cost $p_2$?. There are different possibilities to define this cost function. We will present four alternatives. All of them are based on the idea that shorter and simpler programs generalise better to unseen examples, along the lines of the Occam's razor principle \cite{computationalLearningTheory}. The first three alternatives are evaluated on benchmarks and their effect on performance is discussed in Section~\ref{Eval. Cost functions}.

  \paragraph{nof-nodes}
The first cost function is based only on the number of nodes of the term. It prioritises shorter programs and prefers input variables over library components over holes. It is inductively defined over the terms of the target language as follows.
%
\begin{lstlisting}[style=algorithm]
$\textit{nof-nodes}(c) = 1$
$\textit{nof-nodes}({?x}) = 2$
$\textit{nof-nodes}(i) = 0$
$\textit{nof-nodes}(t_1\; t_2) = 1 + \textit{nof-nodes}(t_1) + \textit{nof-nodes}(t_2)$
$\textit{nof-nodes}(t\; [T]) = 1 + \textit{nof-nodes}(t)$
\end{lstlisting}
%
  \paragraph{nof-nodes-simple-type}
The second cost function adds a factor based on the size of the types appearing in the term. It penalises thus terms with type application depending on the size of the applied types. In particular, arrow types appearing in type applications are heavily penalised. The cost function over types is inductively defined over the quantifier-free subset of the types of the target language.
%
\begin{lstlisting}[style=algorithm]
$\textit{nof-nodes-type}(X) = 1$
$\textit{nof-nodes-type}({?X}) = 0$
$\textit{nof-nodes-type}(I) = 0$
$\textit{nof-nodes-type}(C\; T_1\; \ldots\; T_k) = 0$
$\textit{nof-nodes-type}(T_1 \rightarrow T_2) = 3 + \textit{nof-nodes-type}(T_1) + \textit{nof-nodes-type}(T_2)$

$\textit{nof-nodes-term}(c) = 1$
$\textit{nof-nodes-term}({?x}) = 2$
$\textit{nof-nodes-term}(i) = 0$
$\textit{nof-nodes-term}(t_1\; t_2) = 1 + \textit{nof-nodes-term}(t_1) + \textit{nof-nodes-term}(t_2)$
$\textit{nof-nodes-term}(t\; [T]) = 1 + \textit{nof-nodes-term}(t) + \textit{nof-nodes-type}(T)$

$\textit{nof-nodes-simple-type}(t) = \textit{nof-nodes-term}(t)$
\end{lstlisting}

  \paragraph{no-same-component}
The third cost function additionally penalises terms that use the same component more than once.
%
\begin{lstlisting}[style=algorithm]
$\textit{nof-nodes-type}({?X}) = 3$
$\textit{nof-nodes-type}(I) = 0$
$\textit{nof-nodes-type}(C\; T_1\; \ldots\; T_k)$= $4 + \textit{nof-nodes-type}(T_1) + \ldots + \textit{nof-nodes-type}(T_k)$
$\textit{nof-nodes-type}(T_1 \rightarrow T_2) = 5 + \textit{nof-nodes-type}(T_1) + \textit{nof-nodes-type}(T_2)$

$\textit{nof-nodes-term}(c) = 3$
$\textit{nof-nodes-term}({?x}) = 2$
$\textit{nof-nodes-term}(i) = 0$
$\textit{nof-nodes-term}(t_1\; t_2) = 6 + \textit{nof-nodes-term}(t_1) + \textit{nof-nodes-term}(t_2)$
$\textit{nof-nodes-term}(t\; [T]) = 5 + \textit{nof-nodes-term}(t) + \textit{nof-nodes-type}(T)$

$\textit{count}(t) = \displaystyle \sum_{c_i \text{ appears in } t} \text{(occurrences of } c_1 \text{ in } t) - 1$

$\textit{no-same-component}(t) = \textit{nof-nodes-term}(t) + 3\; \textit{count}(t)$
\end{lstlisting}

  \paragraph{string-length}
The simplest and most imprecise method to take both the number of nodes and the complexity of the types appearing in the term into account is to define the cost of a term as the length of the string representing that term. This method also allows a simple way to weight differently the various library components by choosing a shorter or longer name. However, we decided not to use this cost function for evaluation.

\section{Black list}\label{Black list}
Recall the `replicate' example, where we encountered the superfluous program \lstinline!const [?X] ?$x_1$ ?$x_2$!. The best-first enumeration explores many superfluous branches, such as, for example:
\begin{lstlisting}[style=plain]
foldr [?X] [List ?X] (cons [?X]) (nil [?X]) ?xs
\end{lstlisting}
\begin{lstlisting}[style=plain]
add zero ?n
\end{lstlisting}
Such programs can be ruled out only based on the semantics of the library components. A simple way to prune those superfluous branches is to compile a list of undesired patterns and check each generated program against this list. This is what we call \emph{black list pruning}.

A black list is a list of terms of the target language. Programs that contain a subterm that matches a term from the black list are removed from the candidate programs and their successors are ignored.

The relation \textit{matches} over terms is inductively defined as follows.
\begin{lstlisting}[style=algorithm]
$\textit{matches}({?x},t)$
$\textit{matches}(i,t)$
$\textit{matches}(c,c)$
$\textit{matches}(t_1\; t_2,t_3\; t_4)$ if $\textit{matches}(t_1,t_3)$ and $\textit{matches}(t_2,t_4)$
$\textit{matches}(t_1\; [T_1],t_2\; [T_1])$ if $\textit{matches}(t_1,t_2)$
\end{lstlisting}

As you can see, holes and input variables in the black list match every subterm, a library component matches only itself, a term application matches a term application whose respective left- and right hand sides match and a type application matches a type application if the left hand sides match. Note that the types in a type application are completely ignored.  As a reminder of this semantics, in the following chapters we will typeset black list patterns omitting the type applications and replacing all holes and input variables with the wild-card symbol `\lstinline!_!'.

Pruning based on black lists can be easily integrated in Algorithm~\ref{alg:best-first search}. The result is shown in Algorithm~\ref{alg:blacklist pruning}, where the differences to the original best-first search are highlighted in blue.

\begin{algorithm}
\caption{Best first search with black list\label{alg:blacklist pruning}}
\KwIn{goal type $T$, library components $\Delta$, list of input-output examples $[(\Phi_1, o_1), \ldots , (\Phi_N, o_N)]$, black list $[b_1, \ldots , b_M]$}

queue $\gets$ PriorityQueue.empty compare\\
queue $\gets$ PriorityQueue.push queue $\{\Xi, \Phi_1, \Delta \vdash {?x} :: T\}$\\

\While{not ((PriorityQueue.top queue) satisfies all I/O-examples)}{
	\uIf{{\color{blue}not ((PriorityQueue.top queue) contains subterm from black list)}}{
		successors $\gets$ successor (PriorityQueue.top queue)\\
		queue $\gets$ PriorityQueue.pop queue\\
		\For{all s in successors}{
			queue $\gets$ PrioriryQueue.push queue s\\
		}
	}\Else{
		queue $\gets$ PriorityQueue.pop queue\\
	}
}
\KwOut{PriorityQueue.top queue}
\end{algorithm}

In Section~\ref{Black list generation} we discuss how to synthesise a black list automatically.

\section{Templates}\label{Templates}
In this section we present a slightly different way to explore the search space. The idea is to fix all higher-order components first, producing a \emph{template} for a program, and then fill in the remaining holes with input variables and first-order components. Since programs usually contain only a few higher-order components, the enumeration of templates takes less time than the enumeration of programs of comparable size. Moreover, templates encode well-known patterns of computation and impose meaningful constraints on the remaining holes. Therefore, it should be easy to find the desired program starting from the right template. This allows us to quickly abandon templates if no program satisfying the input-output examples is found within a short timeout.

The library $\Delta$ is split into two contexts: $\Delta_h$, containing all higher-order components, and $\Delta_f$, containing the first-order ones. We also need to introduce a new kind of term: the \emph{delayed hole} $\underline{?x}$. This is a hole, whose instantiation is delayed to the first-order search. The context $\Xi$ binds, additionally to normal holes, delayed holes as well.
A \emph{template} is a term in the target language that may contain delayed holes but does not contain input variables. A template is called \emph{closed} if it does not contain holes (it may, however, contain delayed holes).

The synthesis procedure iterates between two phases: enumeration of templates and, as soon as a closed template is found, enumeration of programs as in Algorithm~\ref{alg:best-first search} using $\Delta_f$ for a limited period of time or until a program satisfying all input-output examples is found.

We additionally restrict the space by requiring a template to have no more than $M$ higher-order components and no more than $P$ delayed holes. Templates are enumerated according to the rules listed below. Those are very similar to the ones defined in Section~\ref{Search space}, except that we cannot instantiate a hole with an input variable but we can delay a hole. All the rules are modified to take into account the restriction on the number of components and the number of closed holes. In order to do this, we need to pass along $m$, the number of higher-order components in the term whose subterms we are traversing.

Analogously to D-VarLib, the rule G-VarLib instantiates a hole with a type application of a higher-order library components to suitable types, if the type of the component unifies with the type of the hole. The rule G-VarDelay delays the instantiation of a hole to the first-order search. G-VarApp replaces a hole with a term application of two fresh holes. G-App expands one of the holes of the template according to one of the three rules listed above. Note that there are no rules to expand a delayed hole.

\begin{prooftree}
\AxiomC{$|\Xi| \leq P$ and $m < M$}
\noLine
\UnaryInfC{$c : \forall X_1.\; \ldots \forall X_n.\; T_c(X_1, \ldots, X_n) \in \Delta_h$}
\noLine
\UnaryInfC{$?X_1, \ldots, ?X_n$ are fresh type holes}
\noLine
\UnaryInfC{$\sigma$ unifies $T$ with $T_c(?X_1, \ldots, ?X_n)$}
\noLine
\UnaryInfC{$\Xi' = \Xi \cup \{{?X_1}, \ldots, {?X_n}\} \setminus \{{?x} : T\}$}
	\RightLabel{G-VarLib}
	\UnaryInfC{$\Xi, \Phi, \Delta_h, m \vdash {?x} :: T \Mapsto \sigma(\Xi'), \Phi, \Delta_h, m+1 \vdash c\; [\sigma(?X_1)]\; \ldots\; [\sigma(?X_n)] :: \sigma(T) $
	}
\end{prooftree}

\begin{prooftree}
\AxiomC{$|\Xi| \leq P$ and $m \leq M$}
	\RightLabel{G-VarDelay}
	\UnaryInfC{$\Xi, \Phi, \Delta_h, m \vdash {?x} :: T \Mapsto \Xi \setminus \{{?x}:T\} \cup \{\underline{?x}:T\}, \Phi, \Delta_h, m \vdash {\underline{?x}} :: T$}
\end{prooftree}

\begin{prooftree}
\AxiomC{$|\Xi| < P$ and $m \leq M$}
\noLine
\UnaryInfC{$?X$ is a fresh type variable}
\noLine
\UnaryInfC{$\Xi' = {\Xi \setminus \{?x:T\} \cup \{{?x_1} : {?X} \rightarrow T, {?x_2} : {?X}, {?X}\}}$}
	\RightLabel{G-VarApp}
	\UnaryInfC{$\Xi, \Phi, \Delta_h, m \vdash {?x} :: T \Mapsto \Xi', \Phi, \Delta_h, m \vdash {?x_1}\;{?x_2} :: T$}
\end{prooftree}


\begin{prooftree}
\AxiomC{$\Xi, \Phi, \Delta_h, m \vdash {?x} :: T_1 \Mapsto \Xi', \Phi, \Delta_h, m' \vdash t_1' :: T_1'$}
	\RightLabel{G-App}
	\UnaryInfC{$\Xi, \Phi, \Delta_h, m \vdash t[{?x}] :: T \Mapsto \Xi', \Phi, \Delta_h, m' \vdash t[t_1] :: [T_1 \mapsto T_1']T$}
\end{prooftree}


%%% Local Variables:
%%% mode: latex
%%% TeX-master: "thesis"
%%% End:
