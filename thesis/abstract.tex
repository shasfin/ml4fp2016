\begin{abstract}


Library components encode well-known computational patterns that human programmers reuse in their programs.  In this thesis, we investigate the benefits of providing a relatively large library of components to a program synthesiser.  We target the synthesis of functional straight-line programs from input-output examples.  Towards this end, we implement a basic synthesis algorithm based on best-first enumeration combined with type-based pruning.   Heuristics are used to guide the search and black lists to prune the search space.  We have evaluated our prototype implementation on simple algorithmic problems over a library of $37$ components.  Results indicate that the performance of our basic algorithm is comparable with that of much more sophisticated state-of-the-art algorithms.

\end{abstract}
