\chapter{Benchmarks} \label{benchmarks}

\note{
Some programs over numbers, some over lists, some over lists of lists and some over trees (What kind of trees?).
For every program, try to get a sample implementation. \\
Types needed: \lstinline?Int, [a], Tree a? \\
Basic components needed:
arithmetic (\lstinline?+, -, *, /?), 
relation (\lstinline?<, <=, ==, /=, >=, >?),
\note{(maybe we do not need relations)}
}

\begin{enumerate}
	\item max of two numbers\\
	(hopefully) the easiest program \\
	\begin{lstlisting}
max :: Int -> Int -> Int
max 0 0 == 0
max 1 0 == 1
max 0 1 == 1
max x y = if x > y then x else y
	\end{lstlisting}
	\note{We don't care about conditionals, we cannot synthesize this.}\\
	This is the only function that requires a conditional branch.
%
	\item square a number
	\begin{lstlisting}
square :: Int -> Int -> Int
square 0 == 0
square 1 == 1
square 2 == 4
square 3 == 9
square x = x * x
	\end{lstlisting}
	That is, basic arithmetic operations like \lstinline!+ - * /! should be provided
%
	\item tetrahedral numbers \\
	\begin{lstlisting}
tetrahedral :: Int -> Int
tetrahedral 1 == 1
tetrahedral 2 == 4
tetrahedral 3 == 10
	\end{lstlisting}
	closed form solution
	\begin{lstlisting}
tetrahedral n = n * (n+1) * (n+2) / 6
	\end{lstlisting}
	iterative solution
	\begin{lstlisting}
tetrahedral n = scanl1 (+) (scanl1 (+) [0..]) !! n
	\end{lstlisting}
	Another iterative solution (without infinite lists)
	\begin{lstlisting}
tetrahedral n = foldl1 (+) (scanl1 (+) (enumFromTo 1 n))
	\end{lstlisting}
	Components needed: \lstinline?scanl1, !!? \\
	Interestingly the iterative version is much faster than the closed form solution
%
	\item prime test \\
	I think this is too difficult
	\begin{lstlisting}
prime :: Int -> Int
prime 1 == 0
prime 2 == 1
prime 3 == 1
prime 4 == 0
prime 25 == 0
prime 29 == 1
prime n = minimum (1 : (map (mod n) (enumFromTo 2 (subtract 1 n))))
	\end{lstlisting}
	Components needed: \lstinline?map, mod, minimum, enumFromTo, subtract?
%
	\item average
	\begin{lstlisting}
average :: [Int] -> Int
average [1] == 1
average [1,3] == 2
average [1,2,3,6] == 6
average xs = (sum xs) `div` (length xs)
	\end{lstlisting}
%
	\item movingAverage (forward)
	\begin{lstlisting}
movingAverage :: Int -> [Int] -> [Int]
movingAverage 1 [1,2,3] == [1,2,3]
movingAverage 2 [1,2,3] == [2,2,3]
movingAverage 3 [3,2,4,1,5,2] == [3,2,3,2,3,2]
movingAverage n xs = map (average . take n) (init $ tails xs)
	\end{lstlisting}
	Components needed: \lstinline?tails? from \lstinline?Data.List? and  \lstinline?average? (one of the benchmarks), as well as \lstinline?map, take? and \lstinline?init? from \lstinline?Prelude?.
%
	\item movingSum (backward)
	\begin{lstlisting}
movingSum :: Int -> [Int] -> [Int]
movingSum 1 [1,2,3] == [1,2,3]
movingSum 2 [1,2,3] == [1,3,5]
movingSum 3 [4,8,6,-1,-2,-3,-1,3,4,5] == [4,12,18,13,3,-6,-6,-1,6,12]
movingSum n xs = scanl1 (+) (zipWith (-) xs (replicate n 0 ++ xs))
	\end{lstlisting}
%
	\item waterflow problem \\
	Given an array of "wall" heights, determine the volume of the puddles that can form if it rains.
	\begin{lstlisting}
water :: [Int] -> Int
water [1,2,3] == 0
water [5,2,5] == 3
water [2,3,1,6,1] == 2
water h = sum $ 
      zipWith (-) 
        (zipWith min (scanl1 max h) (scanr1 max h))
        h
	\end{lstlisting}
%
	\item horner schema to evaluate polynomials
	\begin{lstlisting}
horner :: [Int] -> Int -> Int
horner [1,2,3] 1 == 6
horner [1,2,3] 2 == 11
horner [4,3,2] 3 == 47
horner p x = foldl1 ((+) . (x *)) p
	\end{lstlisting}
	Problem: we do not generate lambda's. Do we generate functions like \lstinline?(x *)??
%
	\item sum-under, sum all integers up to the argument
	\begin{lstlisting}
sum_under :: Int -> Int
sum_under 0 == 0
sum_under 1 == 1
sum_under 2 == 3
sum_under 3 == 6
sum_under 4 == 10
sum_under n = sum [1..n]
	\end{lstlisting}
	Components needed: \lstinline?sum, enumFromTo?
%
	\item factorial \\
	\begin{lstlisting}
factorial :: Int -> Int
factorial 0 == 0
factorial 1 == 1
factorial 3 == 6
factorial 5 == 120
factorial n = product [1..n]
	\end{lstlisting}
	interesting for intermediate states
%
	\item maximum of a list\\
	I don't know (yet) how to specify a "global property" like greater or smaller than all other elements in a list in \textsc{Synquid}. Moreover, it seems a difficult property to extract from input-output examples.
	\begin{lstlisting}
maximum :: [Int] -> Int
maximum [1,3,2] == 3
maximum [4,2,1] == 4
maximum [1,3,5] == 5
maximum xs = foldr max (head xs) xs
	\end{lstlisting}
	Or just use the \lstinline?maximum? function from \lstinline?Prelude?, if it is given as a component
%
	\item append two lists\\
	The specification given by Nadia does not synthesize the usual append function. Maybe it's better to let her know...\\
	Although it's possible to synthesize append in \textsc{Synquid}.
	\begin{lstlisting}
append :: [a] -> [a] -> [a]
append [1,2,3] [4,5,6] == [1,2,3,4,5,6]
append [1,2] [6,2,3] == [1,2,6,2,3]
append xs ys = foldr (:) ys xs
	\end{lstlisting}
	Or use \lstinline?++? from \lstinline?Prelude?, if we decide to provide it for this example too.
%
	\item length of a list \\
	Can be also interesting for intermediate states
	\begin{lstlisting}
length :: [a] -> Int
length [1,2,3] == 3
length [2,2,2] == 3
length [] == 0
length [5] == 1
length xs = sum $ map (const 1) xs
	\end{lstlisting}
%
	\item list reversal
	\begin{lstlisting}
reverse :: [a] -> [a]
reverse [1,2,3] == [3,2,1]
reverse [5,2,3] == [3,2,5]
reverse [6,2,3,1] == [1,3,2,6]
reverse xs = foldl (flip (:)) [] xs
	\end{lstlisting}
%
	\item bagsum: \lstinline![far,bar,gar,bar,bar,far] -> [(bar,3),(far,2),(gar,1)]!\\
	Seems difficult and maybe intermediate states can be helpful. Should I take \lstinline?Int? instead of \lstinline?a?? I mean, \lstinline?a? should belong to the typeclass \lstinline?Ord?, otherwise we cannot yield a sorted output list. And we do not have any other base types anyway. But actually lists of integers are also ordable. Hence also lists of lists of integers and lists of lists of lists of integers and so on.
	\begin{lstlisting}
bagsum :: [a] -> [(a, Int)]
bagsum [1,1,1] == [(1,3)]
bagsum [4,4,2,1,2,1] == [(1,2),(2,2),(4,2)]
bagsum [3,2,1,3,2,3] == [(1,1),(2,2),(3,3)]
bagsum xs = map (head &&& length) (group (sort xs))
	\end{lstlisting}
	Components needed: \lstinline?&&&? from \lstinline?Control.Arrow?. And we also need \lstinline?Tuples? for this one.
%
	\item map \\
	Isn't it a higher order function? I thought we synthesize only first order functions.\\
	How can we provide examples? I mean, we have to write functions as well.\\
	\begin{lstlisting}
map :: (a -> b) -> [a] -> [b]
map f xs = foldr ((:) . f) [] xs
	\end{lstlisting}
%
	\item zipWith \\
	it's a higher order function as well. We need Tuples.\\
	How do we provide the examples?
	\begin{lstlisting}
zipWith :: (a -> b -> c) -> [a] -> [b] -> [c]
zipWith f xs ys = map (uncurry f) (zip xs ys)
	\end{lstlisting}
	Components needed: \lstinline?map, uncurry, zip?.
%
	\item list drop
	\begin{lstlisting}
drop :: Int -> [a] -> [a]
drop 0 [1,2,3] == [1,2,3]
drop 1 [1,2,3] == [2,3]
drop 5 [5,4,2,5] == []
drop 3 [4,2,3,1] == [1]
drop n xs = snd (splitAt n xs)
	\end{lstlisting}
%
	\item droplast, drop the last element of a list
	\begin{lstlisting}
droplast :: [a] -> [a]
droplast [] == []
droplast [1] == []
droplast [1,2,3] == [1,2]
droplast [3,2,1]== [3,2]
droplst = init
droplast' xs = map fst (zip xs (enumFromTo 1 (subtract 1 (length xs))))
droplast'' xs = take (subtract 1 (length xs)) xs
	\end{lstlisting}
%
	\item dropmax, drop the greatest element of a list\\
	$\lambda^2$ takes much more time to synthesize droplast than dropmax. Why?
	\begin{lstlisting}
dropmax :: [Int] -> [Int]
dropmax [1,2] == [1]
dropmax [2,1] == [1]
dropmax [1,2,3] == [1,2]
dropmax [3,2,1] == [2,1]
dropmax [2,3,1] == [2,1]
dropmax xs = filter (/= (maximum xs)) xs
	\end{lstlisting}
%
	\item dedup, remove duplicates from a list \\
	$\lambda^2$ requires more time\\
	\TODO{Find a non-recursive implementation that preserves the order of the elements} Either implement it with \lstinline?groupBy? or say you cannot implement it. Say it's not possible to get the usual semantics of deleting all but the first occurrence of some program.
%
	\item sort by length (on lists of lists)
	\begin{lstlisting}
sortByLength :: [[a]] -> [[a]]
sortByLength [[1,2],[1,2,3],[1]] == [[1],[1,2],[1,2,3]]
sortByLength = sortBy (curry ((uncurry compare) . (length *** length)))
	\end{lstlisting}
	\TODO{Is there a shorter implementation?} What's wrong with this one? \\
	Components needed: \lstinline?(***)? from \lstinline?Control.Arrow?
%
	\item dropmins \\
	$\lambda^2$ required more time to synthesize it
	\begin{lstlisting}
dropmins :: [[Int]] -> [[Int]]
dropmins [[1,2,3],[2,3,1],[2,3],[1]] == [[2,3],[2,3],[3],[]]
dropmins = map dropmin
	\end{lstlisting}
	\TODO{Is there an implementation without the auxiliary function \lstinline?dropmin? and without lambda expressions?} Yes, there is one, but you don't want to see it. With \lstinline?ap? and \lstinline?join?.
%
	\item lasts, last element of every list \\
	another program on nested lists
	\begin{lstlisting}
lasts :: [[a]] -> [a]
lasts [[1,2,3],[3,2],[4,2,4,1],[2]] == [3,2,1,2]
lasts = map last
	\end{lstlisting}
%
	\item member of the tree\\
	Something with trees. Membership seems a difficult thing to learn from input-output examples.
%
	\item count leaves it a tree
%
	\item nodes at level
	The standard Haskell tree is a rose tree. Defined in \lstinline?Data.Tree?.
	
\note{Nadia has more complicated examples with Red-Black-Trees, AVL-trees and different sorting algorithms}
\end{enumerate}




%%% Local Variables:
%%% mode: latex
%%% TeX-master: "thesis"
%%% End:
