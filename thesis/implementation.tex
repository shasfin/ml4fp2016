\chapter{Implementation} \label{ch:implementation}

What could I talk about in this chapter?
\begin{itemize}
\item Programming language and compiler version
\item put the type definitions and explain them (What are Fun, FUN and BuiltinFun) (built-in integers for speed), De-Bruijn indices (make sure you spell the name correctly)
\item Library syntax and the type-checking when added to the library?
\item eager evaluation, describe evaluator
\item Table of implemented components
\end{itemize}


source files
\begin{itemize}
\item b-library.tm: library components defined in the internal language. Built-in components are excluded.
\item library.tm: same as b-library but with additional Scott's integers.
\item builtin.ml: built-in library components and a couple of practical functions to simplify the definition of built-in functions that take multiple arguments
\item lambda.ml: the interpreter. Contains terms and types, evaluation, typing, unification and the generic library lib, which is used for typing and evaluation.
\item library.ml: the library module. Assumes that a library component stores a term definition and a type signature. Has fold and iter. Can return a list of components that unify with a given type, paired with the corresponding substitutions
\item parser and lexer to read \lstinline!@ List #0 -> List #0!.
\item program.ml: a partial program with holes. Encapsulates the notion of fresh hole, hole to be expanded and cost. The cost functions are changed here.
\item syntaxSugar.ml: convenient notation
\item synthesiser.ml: successor rules for best-first search and for templates, the satisfies-all predicate, the three enumeration strategies plain, blacklist and template in two variants, with and without timeout. The versions without timeout output more solutions (depending on the last argument).
\end{itemize}

What is important?





%%% Local Variables:
%%% mode: latex
%%% TeX-master: "thesis"
%%% End:
